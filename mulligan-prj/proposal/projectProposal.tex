\documentclass[a4paper]{article}
\usepackage[letterpaper, margin=1in]{geometry} % page format
\usepackage{listings} % this package is for including code
\usepackage{graphicx} % this package is for including figures
\usepackage{amsmath}  % this package is for math and matrices
\usepackage{amsfonts} % this package is for math fonts
\usepackage{tikz} % for drawings
\usepackage{hyperref} % for urls
\usepackage{stackengine}

\newcommand\tab[1][0.5cm]{\hspace*{#1}}

\title{Project Proposal}
\author{Kaitlyn Mulligan}
\date{2/13/19}

\begin{document}
\lstset{language=Python}

\maketitle

\section{Dog Breed Identification}
When figuring out what topic to do for my project, I decided to explore past kaggle 
competitions to get a few ideas.  I came across a competition about identifying 
different breeds of dogs from images and it sparked my interest.  For this project 
I plan on building a classifier using a Convolutional Neural Network and dog 
breeds.  The past competition on kaggle gives me access to labeled data as well 
as training and testing data.  To build this classifier, I would plan on doing it 
in Tensorflow.  Here is a link to the past kaggle competition which has the data I 
could use to build this classifier: \href{https://www.kaggle.com/c/dog-breed-identification}
{Dog Breed Identification}.  This project would fall under an algorithmic type of project.  
Kaggle is providing me with the problem that they would like to solve.  I will then develop 
a new learning algorithm, or a novel variant of an existing algorithm, to solve it.  This 
project will be difficult as I begin to learn these new topics.  Since it is a previous 
Kaggle competition, there is access to what other people have tried to solve this problem.  
This will be helpful as I begin to teach myself these new ideas and try to create my own 
algorithm to solve this problem.  In addition to these resources, I will need to do some 
research of my own on many of the aspects of this classifier to understand what is happening 
within the algorithm and how to implement it myself.  This project will teach me many new 
things about python and Neural Networks as I have never worked with Neural Networks before 
and have only used python a minimal amount previous to this course.


\end{document}
